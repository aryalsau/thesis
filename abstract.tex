% sherman 1
% $Log: abstract.tex,v $
% Revision 1.1  93/05/14  14:56:25  starflt
% Initial revision
% 
% Revision 1.1  90/05/04  10:41:01  lwvanels
% Initial revision
% 
%
%% The text of your abstract and nothing else (other than comments) goes here.
%% It will be single-spaced and the rest of the text that is supposed to go on
%% the abstract page will be generated by the abstractpage environment.  This
%% file should be \input (not \include 'd) from cover.tex.

% Complex processes in the upper-atmosphere of the Earth give rise to optical emissions in the form of airglow and aurora. Thus, these emissions are key diagnostics to study the processes involved in the upper-atmosphere and specifically the ionosphere-thermosphere (IT,$\sim$ 90-600 km) region. Simultaneous observations of these emissions allows one to (a) determine the characteristics of the precipitating particles that produced the aurora, (b) infer changes in the upper atmosphere and the photochemical processes leading to these emissions. A versatile new ground-based spectrograph known as High-Throughput and Multislit Imaging Spectrograph (HiT\&MIS) has been developed to observe airglow and auroral emissions simultaneously at six selected wavelengths on a round-the-clock basis. The selected spectral features are: HI 656.3 nm, HI 486.1 nm, OI 557.7 nm (green line), OI 630.0 nm (red line), OI 777.4 nm and N$_2^+$ 427.8 nm. 

% For this dissertation, based on multi-spectral measuremets using HiT\&MIS, two questions are addressed. First, (i) how well can the characteristic energy and energy flux of auroral electron precipitation be simultaneously derived using multi-spectral measurements? Second, (ii) did the total solar eclipse of August 21, 2017 cause the brightness perturbation observed in airglow observations?

% During auroral events, the brightnesses of different emissions (e.g., green line, red line, etc) change simultaneously but differently depending on the energy of the precipitating electrons, the density structure of the atmospheric constituents producing these emissions and the mechanisms of how particular emissions are produced. Hence, these simultaneous measurements can be utilized to infer the energy characteristics of the precipitating electrons. Auroral energies have been previously derived by using multi-spectral observations by using ratios of emissions. Auroral fluxes are then derived separately using other methods, once the energy is derived. Auroral emissions can be estimated by utilizing electron transport and chemical reaction models in terms of atmospheric composition, geophysical parameters and the energy and energy flux of the precipitating particles. The method developed in this dissertation derives the energy and flux of auroral electrons by comparing modeled brightnesses with multi-spectral measurements of the same emissions. The results obtained from this method are then compared to methods similar to what has been used before to answer (i).  

% Multi-spectral measurements of airglow can also be used to study and characterize atmospheric waves, known as  Atmospheric Gravity Wave (AGWs), that are produced by various mechanisms in the IT system. These include forcing from the lower atmosphere, heating induced due to increase in ionospheric currents, temperature changes induced by total solar eclipses, etc. Since airglow emissions depend on the neutral structure of the atmosphere and the processes causing AGWs alter the composition of the neutral atmosphere, it causes brightness perturbations in emission features mentioned above. This can be used to derive the wave characteristics of AGWs which manifests asTraveling Ionospheric Disturbances (TIDs) via collisional coupling in the ionosphere. On August 22, 2017, around 8 hours after a total solar eclipse, wavelike brightness perturbation with a dominant period of around 1.5 hours in both red and green lines we observed by HiT\&MIS at Carbondale, IL which was in the path of the totality. From previous studies it has been seen that that eclipse leaves the upper atmosphere disturbed. However; it is not completely understood how long the effect of the eclipse on the upper atmosphere lasts. We derive the the dimensional wave properties of the observed TID, by using the HiT\&MIS measurements plus ionospheric parameters derived from GPS and Ionospheric Sounding measurements. Then we use a Ionosphere-Thermosphere model to estimate the eclipse's effect on the IT system to answer (ii).

Complex processes in the upper atmosphere of the Earth give rise to optical emissions in the form of airglow and aurora. These emissions are diagnostics of photochemical and collisional processes in the upper-atmosphere and specifically the ionosphere-thermosphere (IT, ~ 90-600 km) region. Simultaneous observations of selected emissions allow one to (a) determine the characteristics of the precipitating particles that produce the aurora, b) infer compositional changes in the upper atmosphere. A versatile ground-based spectrograph known as High-Throughput and Multi-slit Imaging Spectrograph (HiT\&MIS) has been developed to observe airglow and auroral emissions simultaneously at six selected wavelengths on a round-the-clock basis. The spectral features used in this thesis are: OI 557.7 nm (green line), OI 630.0 nm (red line), OI 777.4 nm and N$_2^+$ 427.8 nm.  

 

For this dissertation, two questions are addressed. First, can the characteristic energy and energy flux of auroral electron precipitation be simultaneously derived using multispectral measurements? Second, did the total solar eclipse of August 21, 2017 cause the brightness perturbation observed in night-time airglow observations? 

 

During auroral events, the brightnesses of different emissions (e.g., green line, red line, etc) change depending on the energy of the precipitating electrons, the density structure of the atmospheric constituents producing these emissions and the excitation mechanisms of how particular emissions are produced. Hence, simultaneous measurements of optical emissions can be utilized to infer the energy characteristics of the precipitating electrons. Auroral emission characteristics can be modeled by utilizing electron transport and chemical reaction models in terms of atmospheric composition, geophysical parameters and the energy and energy flux of the precipitating particles. The method developed in this dissertation derives the energy and flux of auroral electrons by comparing modeled brightnesses with simultaneous measurements of three emission features. To answer the first question, the results obtained from this method are compared to previous results obtained from previous studies.   

 

Multispectral measurements of airglow can also be used to study and characterize atmospheric waves, known as Atmospheric Gravity Wave (AGWs), that are produced by various mechanisms in the IT system. These include forcing from the lower atmosphere, heating due to increased ionospheric currents, temperature changes induced by total solar eclipses, etc. Airglow emissions depend on the structure of the neutral atmosphere. Since the processes causing AGWs alter the composition of the neutral atmosphere, they cause brightness perturbations in emission features. This relationship can be used to derive the wave characteristics of AGWs which manifest as Traveling Ionospheric Disturbances (TIDs) via collisional coupling in the ionosphere. Approximately 8 hours after the total solar eclipse on August 22, 2017, wavelike brightness perturbations with a dominant period of approximately 1.5 hours were observed in both red and green lines by HiT\&MIS along the path of totality from Carbondale, IL. Previous studies have shown that solar eclipse leaves the upper atmosphere disturbed. However, it is not completely understood how long the effect of the eclipse on the upper atmosphere lasts. In this case, first, the wave properties of the TID derived. Then, measurements from HiT\&MIS and other instruments are compared with an Ionosphere-Thermosphere model to estimate the eclipse's effect on the IT system to answer the second question.
