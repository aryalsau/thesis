% Sherman 1
% Revision 1.1  92/04/22  13:08:20  epeisach

% BE SURE TO READ THE UNIVERSITY'S RULES ON WHAT FIELDS ARE REQUIRED OR ENCOURANGED FOR YOUR DEPARTMENT

\title{Remote sensing of the upper atmosphere using ground-based imaging spectroscopy}

\author{Saurav Aryal}

\prevdegrees{B.S., Winona State University (2009)}
\prevdegrees{M.Sc., Central Michigan University (2011)}

\department{Department of Physics and Applied Physics}

% If the thesis is for two degrees simultaneously, list them both
% separated by \and like this:
% \degree{Doctor of Philosophy \and Master of Science}
\degree{Doctorate}
\degreemonth{January}
\degreeyear{2019}
\thesisdate{January 16, 20189

% If there is more than one supervisor, use the \supervisor command
% once for each.
\supervisor{Supriya Chakrabarti}{Associate Professor}
%\reader{Michelle Scribner-MacLean}{Assistant Professor} % reader is not necessary

% this is the department committee chairman, not the thesis committee chairman. 
\chairman{Robert Giles}{Department Chairman}

% Make the titlepage based on the above information.  If you need
% something special and can't use the standard form, you can specify
% the exact text of the titlepage yourself.  Put it in a titlepage
% environment and leave blank lines where you want vertical space.
% The spaces will be adjusted to fill the entire page.  The dotted
% lines for the signatures are made with the \signature command.
\maketitle

% The abstractpage environment sets up everything on the page except
% the text itself.  The title and other header material are put at the
% top of the page, and the supervisors are listed at the bottom.  A
% new page is begun both before and after.  Of course, an abstract may
% be more than one page itself.  If you need more control over the
% format of the page, you can use the abstract environment, which puts
% the word "Abstract" at the beginning and single spaces its text.

%% You can either \input (*not* \include) your abstract file, or you can put
%% the text of the abstract directly between the \begin{abstractpage} and
%% \end{abstractpage} commands.

% First copy: start a new page, and save the page number.
\newpage
\thispagestyle{empty}
\mbox{}
\newpage
\thispagestyle{empty}

% Second copy: start a new page, and reset the page number.  This way,
% the second copy of the abstract is not counted as separate pages.
\pagestyle{plain}
\newpage
\setcounter{page}{1}
\pagenumbering{roman}
\begin{abstractpage}
% sherman 1
% $Log: abstract.tex,v $
% Revision 1.1  93/05/14  14:56:25  starflt
% Initial revision
% 
% Revision 1.1  90/05/04  10:41:01  lwvanels
% Initial revision
% 
%
%% The text of your abstract and nothing else (other than comments) goes here.
%% It will be single-spaced and the rest of the text that is supposed to go on
%% the abstract page will be generated by the abstractpage environment.  This
%% file should be \input (not \include 'd) from cover.tex.

Complex processes in the upper-atmosphere of the Earth give rise to optical emissions in the form of airglow and aurora. Thus, these emissions are key diagnostics to study the processes involved in the upper-atmosphere. Simultaneous observations of these emissions allows one to (a) determine the characteristics of the precipitating particles that produced the aurora, (b) infer changes in the upper atmosphere and the photochemical processes leading to these emissions. A versatile new ground-based spectrograph known as High-Throughput and Multislit Imaging Spectrograph (HiT\&MIS) has been developed to observe airglow and auroral emissions simultaneously at six selected wavelengths on a round-the-clock basis. The selected spectral features are: HI 656.3 nm, HI 486.1 nm, OI 557.7 nm, OI 630.0 nm, OI 777.4 nm and $N_2$$^+$ 427.8 nm. The brightnesses of these emission features change simultaneously but differently depending on the energy of the precipitating electrons, the density structure of the atmospheric constituents producing these emissions and the mechanisms of how particular emissions are produced. Hence, these simultaneous measurements can be utilized to infer the energy characteristics of the precipitating electrons. For the current dissertation, (i) the characteristic energy and energy flux of energetic electron precipitation were inferred based on simultaneous measurements of OI 557.7 nm, OI 630.0 nm and $N_2$$^+$ 427.8 nm emission features during an auroral events. Measurement of these optical emissions were modeled based on an electron transport and chemical reaction model, in terms of atmospheric composition, geophysical parameters and the energy and energy flux of the precipitating particles.The method presented in this dissertation provides a way multispectral imagers can be used as a stand-alone tool to derive energy and energy flux during auroral events. In addition, (ii) because each airglow emission occurs at a particular altitude, round-the-clock measurement of multiple airglow emission features will be used to trace changes in the upper atmosphere at different altitudes. The proposed study will allow us to observe the solar-terrestrial interactions from the ground at high temporal resolution during both geomagnetically quiet and active times.
\end{abstractpage}

%%%%%%%%%%%%%%%%%%%%%%%%%%%%%%%%%%%%%%%%%%%%%%%%%%%%%%%%%%%%%%%%%%%%%%
% -*-latex-*-



